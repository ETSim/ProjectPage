\chapter{Résultats}

\section{Résultats du simulateur de friction 3D}
\lipsum[1]

\section{Analyse de données réelles}
\lipsum[1]

\section*{Calcul de la normale à partir de 6 images – Application sur le banc GelSight}

Dans notre système GelSight, un capteur tactile optique de haute précision, nous utilisons une méthode simple pour reconstruire la carte des normales d'une surface. Le banc GelSight capture 6 images de la même surface, chacune éclairée par une source de lumière positionnée différemment. Pour chaque pixel de l'image, la valeur d'intensité représente la quantité de lumière réfléchie. Ces 6 intensités, notées \( I_1, I_2, \dots, I_6 \), nous aident à déduire la direction de la normale de la surface à cet endroit.

\vspace{0.5em}
\textbf{Étape 1 : Définition des directions lumineuses}\\
Chaque image est prise sous un angle \(\theta_i\) connu. Pour simplifier, on représente la direction de la lumière sous forme d'un vecteur :
\[
\mathbf{L}_i = \begin{pmatrix} \cos\theta_i \\ \sin\theta_i \\ 1 \end{pmatrix}
\]
pour \( i = 1, \dots, 6 \).

\vspace{0.5em}
\textbf{Étape 2 : Modélisation du problème}\\
La lumière réfléchie par la surface à chaque pixel dépend du produit scalaire entre la normale de la surface 
\[
\mathbf{N} = \begin{pmatrix} N_x \\ N_y \\ N_z \end{pmatrix}
\]
et la direction de la lumière \(\mathbf{L}_i\). Ainsi, on a pour chaque image :
\[
I_i = \mathbf{N} \cdot \mathbf{L}_i \quad \text{pour } i=1,\dots,6.
\]
En regroupant ces 6 équations, on peut écrire :
\[
\begin{pmatrix}
I_1 \\ I_2 \\ \vdots \\ I_6
\end{pmatrix}
=
\begin{pmatrix}
\cos\theta_1 & \sin\theta_1 & 1\\[1mm]
\cos\theta_2 & \sin\theta_2 & 1\\[1mm]
\vdots & \vdots & \vdots\\[1mm]
\cos\theta_6 & \sin\theta_6 & 1
\end{pmatrix}
\begin{pmatrix}
N_x \\ N_y \\ N_z
\end{pmatrix}.
\]
On note cela succinctement :
\[
I = L\, \mathbf{N}.
\]

\vspace{0.5em}
\textbf{Étape 3 : Calcul de la normale}\\
Pour déterminer la normale \(\mathbf{N}\), on résout le système par la méthode de la pseudo-inverse. On calcule l'inverse généralisée de \(L\) :
\[
L^+ = (L^\top L)^{-1} L^\top,
\]
ce qui nous permet d'obtenir :
\[
\mathbf{N} = L^+ \, I.
\]
Ensuite, on normalise la normale pour obtenir une vecteur unitaire :
\[
\mathbf{N}_{\text{normalisée}} = \frac{\mathbf{N}}{\|\mathbf{N}\|}.
\]

\vspace{0.5em}
\textbf{Exemple de code en Python}

\begin{lstlisting}
import numpy as np

# Les 6 angles (en degrés) des sources de lumière
angles_deg = [330, 30, 90, 120, 180, 270]

# Conversion en radians
angles = np.radians(angles_deg)

# Construction de la matrice des directions lumineuses
L = np.array([[np.cos(a), np.sin(a), 1.0] for a in angles])

# Pour un pixel, supposons que nous avons les intensités mesurées suivantes :
I = np.array([I1, I2, I3, I4, I5, I6])  # Remplacez I1, I2, ..., I6 par les valeurs mesurées

# Calcul de la pseudo-inverse de L
L_pinv = np.linalg.pinv(L)

# Calcul de la normale
N = L_pinv @ I

# Normalisation de la normale
N_normalized = N / np.linalg.norm(N)

print("La normale (unitaire) au pixel est :", N_normalized)
\end{lstlisting}

\vspace{0.5em}
\textbf{Résumé pour le banc GelSight :}
\begin{enumerate}
  \item \textbf{Acquisition des images :} Le banc GelSight prend 6 images d'une surface avec des éclairages différents.
  \item \textbf{Mesure de l'intensité :} Chaque image fournit la luminosité \(I_i\) à chaque pixel.
  \item \textbf{Modélisation :} On considère que \(I_i\) est proportionnel au produit scalaire entre la normale \(\mathbf{N}\) et la direction de lumière \(\mathbf{L}_i\).
  \item \textbf{Résolution :} En regroupant les 6 équations, on obtient \( I = L\, \mathbf{N} \). On résout ce système avec la pseudo-inverse pour obtenir \(\mathbf{N}\).
  \item \textbf{Normalisation :} Enfin, on normalise \(\mathbf{N}\) pour obtenir une normale unitaire.
\end{enumerate}

Cette méthode simple nous permet de reconstruire la \emph{normal map} de la surface mesurée par le système GelSight.

\section{Analyse de surface imprimée en 3D}
\lipsum[1]
