\chapter{Revue de la Litt\'erature}

\section{Catégories de Frottement}

Le frottement, force s'opposant au mouvement relatif entre deux surfaces en contact, se divise en plusieurs catégories selon les conditions et les mécanismes d'interaction :

\begin{itemize}
    \item \textbf{Frottement sec (\textit{Dry Friction}) :}  
    Se produit entre deux surfaces solides sans lubrification et se subdivise en :
    \begin{itemize}
        \item \textit{Frottement statique :} Résiste au début du mouvement relatif. La force maximale avant le mouvement est \(F_t = \mu_s F_N\), où \(\mu_s\) est le coefficient de frottement statique.
        \item \textit{Frottement cinétique :} Agit pendant le mouvement relatif, décrit par \(F_t = \mu_k F_N\), avec \(\mu_k < \mu_s\).
    \end{itemize}

    \item \textbf{Frottement lubrifié (\textit{Lubricated Friction}) :}  
    Implique un fluide (huile, eau, gaz) entre les surfaces, réduisant le frottement et l'usure. Trois régimes principaux :
    \begin{itemize}
        \item \textit{Lubrification hydrodynamique :} Les surfaces sont totalement séparées par une couche de fluide.
        \item \textit{Lubrification mixte :} Le contact se fait partiellement à travers le fluide et partiellement via les aspérités.
        \item \textit{Lubrification limite :} Un film de fluide très fin est présent, les aspérités dominent l'interaction.
    \end{itemize}

    \item \textbf{Frottement roulant (\textit{Rolling Friction}) :}  
    Se produit lorsqu’un objet roule sur une surface, comme une roue ou une sphère. La résistance est généralement moindre que celle du frottement sec et dépend de :
    \begin{itemize}
        \item La déformation des surfaces de contact.
        \item Les propriétés des matériaux (module d'élasticité, rugosité).
        \item La force normale appliquée.
    \end{itemize}

    \item \textbf{Frottement fluide (\textit{Fluid Friction}) :}  
    Résistance au mouvement relatif d’un corps solide dans un fluide (liquide ou gaz), souvent modélisée en mécanique des fluides :
    \begin{itemize}
        \item \textit{Écoulement laminaire :} Frottement proportionnel à la vitesse relative.
        \item \textit{Écoulement turbulent :} Frottement proportionnel au carré de la vitesse relative.
    \end{itemize}
    Les lois de Stokes et de Navier-Stokes sont couramment utilisées pour quantifier ce type de frottement.
\end{itemize}



\subsection{Applications et Implications}

L'utilisation des modèles fractals pour décrire les surfaces et les propriétés tribologiques présente plusieurs avantages :
\begin{itemize}
    \item \textbf{Précision accrue :} Représentation fidèle des structures multi-échelles des surfaces.
    \item \textbf{Prédiction améliorée :} Meilleure compréhension des relations entre rugosité, adhérence et frottement.
    \item \textbf{Optimisation des surfaces :} Conception de surfaces adaptées à des applications industrielles spécifiques, telles que les interfaces à faible frottement ou à haute adhérence.
\end{itemize}

L'intégration de ces modèles dans des simulations tribologiques permet d'optimiser des composants critiques dans des secteurs comme l'aérospatial, l'automobile et les dispositifs médicaux.

\section{Approches de Modélisation du Frottement}

\subsection{Th\'eorie de Contact Hertzien}
Utilisée pour modéliser les contacts à petite échelle entre surfaces courbées, telles que des sphères.

\subsection{Modèles de Contact Local}
Considèrent les forces locales au sein des zones de contact, permettant une analyse détaillée des interactions à petite échelle.

\section{Lois de Coulomb}

Les lois de Coulomb définissent le frottement comme proportionnel à la force normale appliquée, distinguant le frottement statique et cinétique.

\subsection{Formule de Coulomb}
\begin{equation}
F_t = \mu F_N \label{eq:coulomb}
\end{equation}
où \(\mu\) est le coefficient de frottement, dépendant des propriétés des matériaux et de la géométrie de l'interface. Bien que cette loi ne distingue pas explicitement entre frottement statique et cinétique, il est empiriquement établi que \(\mu_s \geq \mu_k\), où \(\mu_s\) est le coefficient de frottement statique et \(\mu_k\) celui cinétique.

\section{Tribologie}

La tribologie étudie les interactions entre surfaces en mouvement, incluant le frottement, l'usure et la lubrification. Elle est essentielle dans la conception de matériaux et de dispositifs mécaniques performants et durables.

\section{Instruments de Mesure du Frottement}

Pour quantifier le frottement, divers instruments sont utilisés :
\begin{itemize}
    \item \textbf{Tribomètre :} Mesure directe des coefficients de frottement.
    \item \textbf{Microscope à Force Atomique (AFM) :} Réalise des mesures à l'échelle atomique.
    \item \textbf{Tests de Glissement :} Évaluent le frottement statique et cinétique sur différentes surfaces.
\end{itemize}

\subsection{Frottement des \'{E}lastomères}
Utile pour modéliser les matériaux souples tels que le caoutchouc, en évaluant leurs propriétés de frottement spécifiques.

\subsection{Microscopie à Force Atomique (AFM)}
Permet de mesurer les forces à l'échelle atomique, offrant une compréhension détaillée de la nature moléculaire du frottement.

\section{Méthodes de Calcul du Frottement selon l'Échelle}

Le calcul du frottement varie considérablement selon l'échelle d'étude, chaque échelle nécessitant des approches spécifiques adaptées aux phénomènes dominants.

\subsection{Échelle Macroscopique}
\label{subsec:macroscopique}

\subsubsection{Modèles Empiriques}
Définis principalement par les lois de Coulomb, utilisant des coefficients de frottement statique (\(\mu_s\)) et cinétique (\(\mu_k\)) déterminés expérimentalement. Simples à implémenter, ils sont couramment utilisés dans les logiciels de simulation mécanique pour prédire les forces de frottement dans des systèmes complexes.

\subsubsection{Simulations Numériques}
Intègrent les lois de Coulomb dans des environnements de simulation tels que MATLAB, ANSYS ou COMSOL, permettant de prédire les forces de frottement en fonction des conditions de charge et des mouvements relatifs des objets.

\subsubsection{Analyse Dimensionnelle}
Estime les forces de frottement en fonction des dimensions et des matériaux des objets en contact, basée sur des règles de similarité, sans nécessiter de modèles complexes.

\subsection{Échelle Mésoscopique}
\label{subsec:mesoscopique}

\subsubsection{Modèles Basés sur les Asperités}
Simulent les interactions entre asperités individuelles, prenant en compte leur distribution, forme et densité pour estimer les forces de frottement globales.

\subsubsection{Méthodes de Monte Carlo}
Évaluent les contributions statistiques des asperités irrégulières à la force de frottement, en simulant de multiples interactions pour obtenir une estimation prédictive du frottement moyen.

\subsubsection{Approches Continuum-Micro}
Combinaison de modèles continus pour les grandes échelles et de modèles microscopiques pour les asperités locales, offrant une prédiction précise des forces de frottement dans des systèmes complexes.

\subsection{Échelle Microscopique}
\label{subsec:microscopique}

\subsubsection{Simulation de Dynamiques Moléculaires (MD)}
Modélisent les interactions entre atomes et molécules pour comprendre les mécanismes de frottement au niveau microscopique, observant déformations, rotations et glissements des asperités sous différentes conditions de charge.

\subsubsection{Modèles Basés sur la Théorie des Joints}
Considèrent les contacts ponctuels entre surfaces et les forces résultantes à ces joints, permettant de déterminer les forces de frottement en fonction des contraintes locales et des déformations des joints de contact.

\subsubsection{Méthodes Finies}
Utilisent des éléments finis pour simuler les déformations locales des surfaces en contact, prédisant comment ces déformations influencent les forces de frottement.

\subsection{Échelle Atomique}
\label{subsec:atomique}

\subsubsection{Calculs de la Théorie de la Fonctionnelle de la Densité (DFT)}
Calculent les interactions électroniques entre atomes et molécules en contact, offrant une compréhension détaillée des forces de frottement en tenant compte des effets quantiques.

\subsubsection{Modèles Atomistiques}
Simulent les positions et les mouvements des atomes pour étudier les mécanismes fondamentaux du frottement, visualisant les interactions atomiques et les processus de glissement au niveau atomique.

\subsubsection{Méthodes Ab Initio}
Basées sur des principes physiques fondamentaux sans paramétrage empirique, elles fournissent une compréhension détaillée des forces de frottement au niveau atomique, bien qu'elles soient limitées par des exigences computationnelles élevées.

\section{Synthèse des Méthodes de Calcul du Frottement}

Chaque méthode de calcul du frottement présente des avantages et des limitations selon l'échelle d'étude. À l'échelle macroscopique, les modèles empiriques sont simples mais peuvent manquer de précision pour des configurations complexes. Les approches mésoscopiques offrent une meilleure compréhension des interactions locales, nécessitant toutefois des calculs plus intensifs. À l'échelle microscopique et atomique, les simulations de dynamiques moléculaires et les calculs DFT fournissent des détails précis, mais sont limitées par leurs exigences computationnelles et les approximations nécessaires pour traiter de grands systèmes.

L'intégration de ces méthodes à différentes échelles permet d'obtenir une vision complète des phénomènes de frottement, facilitant le développement de modèles multiscales capables de prédire avec précision les forces de frottement dans une variété d'applications industrielles et scientifiques.

\subsection{Tableau Comparatif des Méthodes de Calcul du Frottement}

Pour mieux comprendre les différences et complémentarités entre les méthodes de calcul du frottement à diverses échelles, le tableau comparatif suivant est présenté.

\begin{table}[h]
    \centering
    \begin{tabular}{|l|l|l|l|}
        \hline
        \textbf{Échelle} & \textbf{Méthodes} & \textbf{Avantages} & \textbf{Limitations} \\ \hline
        Macroscopique & Modèles empiriques, Simulations numériques & Simples à implémenter, Applicables à grande échelle & Précision limitée pour configurations complexes \\ \hline
        Mésoscopique & Modèles basés sur les asperités, Monte Carlo, Continuum-Micro & Compréhension détaillée des interactions locales & Calculs intensifs, Nécessitent des données détaillées \\ \hline
        Microscopique & Dynamiques moléculaires (MD), Théorie des joints, Méthodes finies & Détails précis des interactions locales & Exigences computationnelles élevées \\ \hline
        Atomique & Calculs DFT, Modèles atomistiques, Méthodes ab initio & Compréhension fondamentale des forces atomiques & Limités à petits systèmes, Très coûteux en calcul \\ \hline
    \end{tabular}
    \caption{Comparaison des méthodes de calcul du frottement selon l'échelle.}
    \label{tab:comparaison_methodes_frottement}
\end{table}

\section{Microfacettes et Types de Frottement}

Les microfacettes jouent un rôle crucial dans la détermination des propriétés de frottement des surfaces. Elles permettent de modéliser de manière détaillée les interactions locales entre les asperités des surfaces en contact. Cette section détaille différents types de frottement basés sur l'orientation et la distribution des microfacettes.

\begin{table}[h]
    \centering
    \renewcommand{\arraystretch}{1.3}
    \begin{tabular}{|l|p{4.5cm}|p{5cm}|p{4cm}|}
        \hline
        \textbf{Type de Frottement} & \textbf{Définition} & \textbf{Caractéristiques Clés} & \textbf{Exemple} \\
        \hline
        \textbf{Frottement Isotrope} & Même frottement dans toutes les directions. & - Résistance uniforme \\ - Modèle simple & Un bloc de bois glissant sur une table lisse. \\
        \hline
        \textbf{Frottement Anisotrope} & Frottement qui varie selon la direction du glissement. & - Importance de la direction \\ - Plus réaliste pour certains matériaux & Un morceau de tissu glissé sur une surface : plus facile le long du tissage, plus difficile contre le tissage. \\
        \hline
        \textbf{Frottement Orthotrope} & Forme spécifique d’anisotropie avec deux directions principales. & - Deux directions clés \\ - Différences symétriques & Un tissu tissé : plus facile dans la longueur, plus difficile dans la largeur. \\
        \hline
        \textbf{Frottement Entièrement Anisotrope} & Frottement qui peut varier dans toutes les directions. & - Toutes les directions sont uniques \\ - Contrôle détaillé & Un tapis à motifs où le glissement en diagonale diffère du glissement droit. \\
        \hline
        \textbf{Frottement Hétérogène} & Frottement qui change selon l’endroit sur la surface. & - Varie à travers la surface \\ - Causé par des changements de texture ou d’usure & Une route avec des plaques de glace et du revêtement sec. \\
        \hline
        \textbf{Frottement Asymétrique} & Frottement différent selon la direction du glissement. & - Différences directionnelles \\ - Causé par des caractéristiques de surface & Une manche en fourrure : plus facile à glisser dans un sens que dans l’autre. \\
        \hline
    \end{tabular}
    \caption{Résumé des différents types de frottement}
    \label{tab:frottement}
\end{table}


\section{Fractalit\'e et Rugosit\'e de Surface}

La théorie fractale, introduite par Benoît Mandelbrot, établit un lien entre la rugosité de surface et la dimension fractale, permettant de contrôler les propriétés des matériaux et la formation de copeaux lors de l'usinage. Toutefois, les fractales ne peuvent représenter une surface usinée à toutes les échelles, car elles négligent la géométrie de la lame de coupe \cite{davim2010surface, bez2011duality}.

Les descripteurs fractals des surfaces sont cruciaux pour corréler les propriétés physiques des surfaces avec leur structure. Dans divers domaines, il est souvent difficile de relier le comportement physique, électrique et mécanique aux descripteurs conventionnels de rugosité ou de forme de surface. En combinant des mesures de fractalit\'e avec celles de rugosité ou de forme, certains phénomènes interfaciaux, tels que la rigidité du contact et la résistance au frottement électrique, peuvent être mieux interprétés en relation avec la structure de surface \cite{davim2010surface, bez2011duality}.

\section{Modèles de Frottement}

\subsection{Origine du Frottement}
La tribologie, science des contacts de surfaces en mouvement, englobe le frottement, la lubrification, l’usure et l’adhésion. Une composante clé est la dissipation d’énergie par la rupture de liaisons moléculaires, la déformation plastique et l’usure. Le frottement, force opposée au mouvement tangentiel entre deux surfaces, est central dans ce domaine.

À l’échelle atomique, le frottement dépend de la vitesse de contact \(v_c\) et de la surface réelle de contact \(A_c\), exprimé par :
\begin{equation}
F_t = \eta v_c A_c \label{eq:frottement_atomique}
\end{equation}
où \(\eta = \frac{\rho}{\tau}\), \(\rho\) étant la densité massique surfacique et \(\tau\) un temps caractéristique de glissement. À cette échelle, la force de frottement dépend de l’aire de contact réelle et est indépendante de la force normale appliquée.

\subsection{Adhésion de Bowden et Tabor}
Le modèle classique de Bowden et Tabor explique le frottement en termes d’adhésion et de déformation plastique des asperités. Ils postulent que la vraie aire de contact \(A_c\) est proportionnelle à la force normale \(F_N\), et que la force de frottement totale \(F_t\) est liée à la résistance au cisaillement des matériaux au niveau des asperités :
\begin{equation}
\mu_k = \frac{\sigma}{E^*} \label{eq:bowden_tabor}
\end{equation}
où \(\sigma\) est la résistance au cisaillement du matériau plus mou et \(E^*\) le module de Young effectif. Ce modèle prédit avec précision les coefficients de frottement dans les contacts métal-métal et est largement utilisé en industrie.

\subsection{Modèle LuGre}
Le modèle de LuGre est un modèle avancé pour la simulation du frottement dans les systèmes dynamiques, intégrant les phénomènes de frottement statique et dynamique via la déformation élastique des microstructures à l’interface des surfaces. Il s’exprime par :
\begin{equation}
\dot{z} = v - |v| g(v) z \label{eq:luGre_z}
\end{equation}
où \(z\) est la variable d’état décrivant l’évolution des asperités, \(v\) la vitesse relative des surfaces, et \(g(v)\) une fonction dépendant de la vitesse ajustant l’amplitude du frottement selon le régime (adhérence ou glissement).

La force de frottement \(F_t\) est donnée par :
\begin{equation}
F_t = \sigma_0 z + \sigma_1 \dot{z} + \sigma_2 v \label{eq:luGre_Ft}
\end{equation}
avec \(\sigma_0\), \(\sigma_1\), et \(\sigma_2\) des paramètres ajustables dépendant des propriétés du matériau. Ce modèle permet une transition fluide entre les régimes statique et dynamique, et est largement utilisé dans la simulation des systèmes mécaniques avec frottement.

\subsection{Modèle Stick-Slip et relation \(F - v\)}
Le modèle Stick-Slip (adhérence-glissement) est essentiel pour comprendre le comportement des surfaces en contact lorsque la vitesse relative varie. Il simule les oscillations de frottement observées dans de nombreux systèmes mécaniques, tels que les freins ou les moteurs.

Une loi empirique décrivant la relation force-vitesse (\(F - v\)) est souvent utilisée :
\begin{equation}
F_t = \mu_s (1 - e^{-\beta v}) \label{eq:stick_slip}
\end{equation}
où \(\mu_s\) est le coefficient de frottement statique et \(\beta\) caractérise la transition entre adhérence et glissement.

\subsection{Modèle d’Alart-Curnier}
Ce modèle utilise une approche variationnelle pour résoudre les problèmes de contact avec frottement de manière non linéaire, basé sur un principe de complémentarité similaire à celui utilisé en optimisation.

Les conditions de contact sont exprimées via des multiplicateurs de Lagrange, et la friction est modélisée par une inégalité de complémentarité non lisse. L’algorithme d’Alart-Curnier résout itérativement les forces normales et tangentielles en tenant compte de la loi de Coulomb, étant particulièrement adapté aux simulations mécaniques complexes avec des comportements anisotropes ou non linéaires.