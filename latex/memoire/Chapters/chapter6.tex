\chapter{Conclusion}

\section{Limitations}
Malgr\'e les succ\`es obtenus dans cette recherche, plusieurs limitations doivent \^{e}tre signal\'ees :
\begin{itemize}
    \item \textbf{Pr\'ecision des Surfaces Imprim\'ees en 3D :} Les impressions 3D ont pr\'esent\'e des variations inattendues dans les surfaces g\'en\'er\'ees, ce qui a rendu difficile le calcul des valeurs de frottement exactes. Ces d\'efauts peuvent \^{e}tre dus \`a la qualit\'e des imprimantes ou des mat\'eriaux utilis\'es.
    \item \textbf{Scans 3D :} Les scanners 3D n'ont pas toujours r\'eussi \`a capturer les d\'etails microscopiques critiques, limitant la pr\'ecision des mod\`eles de simulation.
    \item \textbf{Donn\'ees de R\'ef\'erence :} Les valeurs de frottement trouv\'ees dans la litt\'erature et en ligne manquent souvent de pr\'ecision ou de coh\'erence. Ces impr\'ecisions compliquent la validation des simulations.
    \item \textbf{Mat\'eriaux Plus \'Elastiques :} Les simulations et tests r\'eels ont principalement port\'e sur des mat\'eriaux rigides, laissant un manque notable dans l'\'etude des mat\'eriaux plus \'elastiques tels que le caoutchouc ou les polym\`eres.
    \item \textbf{Calculs en Temps R\'eel :} L'absence d'une capacit\'e de calcul en temps r\'eel des coefficients de frottement et des param\`etres li\'es, comme la chaleur, la propret\'e, l'abrasion et l'humidit\'e, constitue une limite importante dans le cadre de cette recherche.
    \item \textbf{Mod\'elisation G\'en\'erale :} Le mod\`ele utilis\'e est g\'en\'eral et ne prend pas en compte les variations des coefficients de frottement selon le contexte sp\'ecifique. La tribologie est une science complexe, et les coefficients peuvent varier de mani\`ere significative en fonction des conditions environnementales et mat\'erielles.
\end{itemize}

\section{Future Work}
Pour surmonter les limitations identifi\'ees et approfondir la recherche, plusieurs pistes d'am\'elioration sont envisag\'ees :
\begin{itemize}
    \item \textbf{Am\'elioration des Surfaces 3D :} Utiliser des imprimantes 3D de qualit\'e sup\'erieure et des mat\'eriaux plus homog\`enes pour r\'eduire les variations dans les surfaces imprim\'ees.
    \item \textbf{Techniques de Scannage Avanc\'ees :} Int\'egrer des scanners 3D de pr\'ecision plus \'{e}lev\'ee ou des m\'ethodes de scannage alternatives pour capturer des d\'etails microscopiques plus fiables.
    \item \textbf{Base de Donn\'ees de R\'ef\'erence :} Cr\'eer une base de donn\'ees consolid\'ee avec des valeurs de frottement valides et reproductibles pour une gamme vari\'ee de mat\'eriaux.
    \item \textbf{\'Etude des Mat\'eriaux \'Elastiques :} \'Etendre la m\'ethodologie pour inclure des tests sur des mat\'eriaux \'elastiques, en particulier ceux couramment utilis\'es dans les applications industrielles comme le caoutchouc et les polym\`eres.
    \item \textbf{Calculs en Temps R\'eel :} D\'evelopper une capacit\'e de simulation en temps r\'eel pour calculer des param\`etres dynamiques comme la chaleur, l'abrasion et l'humidit\'e, qui influencent significativement le frottement.
    \item \textbf{Contextualisation des Mod\'eles :} Adapter les mod\`eles \`a des sc\'enarios sp\'ecifiques pour mieux repr\'esenter les variations des coefficients de frottement en fonction des conditions environnementales et mat\'erielles.
    \item \textbf{Exploration de Nouveaux Param\`etres :} Int\'egrer des param\`etres suppl\'ementaires comme la r\'esistance chimique et la dur\'ee de vie des mat\'eriaux sous des contraintes tribologiques.
\end{itemize}

En int\'egrant ces am\'eliorations, nous pourrions non seulement r\'eduire les erreurs existantes mais \'{e}galement ouvrir la voie \`a des applications plus vari\'ees et pr\'ecises dans les domaines industriels et scientifiques.
