\begin{avantpropos}

    Le phénomène de friction est d'une grande complexité, émergeant de la combinaison de déformations élastiques et plastiques, ainsi que d'interactions moléculaires à l'interface de contact. Lorsque deux surfaces entrent en contact, leur rugosité et la force normale appliquée déterminent la zone réelle d'interaction, régissant ainsi les processus de déformation et d'adhésion moléculaire. À l'échelle macroscopique, ce contact se traduit par une multitude de microscopiques points de contact, chacun contribuant aux forces qui génèrent le frottement global.
    
    Dans cette thèse, nous nous intéressons à la simulation de la friction dans des conditions où le contact est non lubrifié et où les déformations élastiques jouent un rôle clé au niveau des aspérités des surfaces. Afin de mieux comprendre et prédire le comportement frictionnel, nous avons collecté un vaste ensemble de données en faisant varier plusieurs paramètres influençant la friction entre deux surfaces. Nous avons ensuite développé un modèle approximatif exploitant les corrélations observées dans ces données, offrant ainsi une alternative à la fois polyvalente et plus précise que les tables de coefficients de friction traditionnellement utilisées dans les systèmes physiques.
    
    Ce modèle, plus efficient en termes de calcul, vise à fournir des prédictions fiables et à élargir notre compréhension du phénomène de friction dans divers contextes d'application industrielle et technologique.
    
    \end{avantpropos}